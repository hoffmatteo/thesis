\chapter{Introduction}
\label{cha:Chapter1_Introduction}

\iffalse

Total length: up to 5 Months = ~20 weeks oder
\selectlanguage{german}{

Abstract fertig machen, konkreter werden --> vor allem Methoden, kleiner Ansatz


Length: 1-2 pages
Inhaltsverzeichnis --> auch mehr Punkte

Methodology and Implementation --> kann man auch anders aufteilen
Motivation, Hintergrund

Warum diese Plattform --> einfache API

Auch aufpassen illegal

Wann schreiben? --> am Anfang, am Ende? Erstmal Entwurf am Ende
}
\selectlanguage{english}
%
%Effort: 1-2 days
%\begin{itemize}
%\item Major importance of social media, impact on politics, source of information
%\item Massive number of posts allows for a good basis for analysis
%\item Analyse current trends, public's opinion on certain issues, news and other topics
%\item Motivation: Condense massive amount of information/opinions into an intelligible format
%\item Chosen platform: Twitter, short messages <280 characters, massive user base, broad topic range
%\end{itemize}
%

\fi

Social media has grown to be one of the most important part of many people's daily lives. While most people know the most popular social media platforms such as Facebook, Twitter, Instagram, and more, Kaplan and Haenlein give a more specific definition. They define Social Media as "a group of Internet-based applications that build on the ideological and technological foundations of Web 2.0, and that allow the creation and exchange of User Generated Content" \cite[p.~61]{KAPLAN201059}. 

According to the Pew Research Center, about 72\% of Americans use any kind of social media, a number that is even higher if only young people are considered. Additionally, the majority of users use certain platforms daily, sometimes even multiple times a day, which shows how indispensable social media has become for a lot of people \cite{pew:socialmedia}. The effect of this high usage can be seen in many areas of public and personal life. One interesting area is politics. In the 2016 U.S presidential election, the candidate's social media post were more often used as a source for information compared to more traditional online mediums, such as their website and email \cite{pew:2016source}. In 2020, 23\% of U.S social media users stated that social media led them to change their opinion on an issue \cite{pew:2020influence}. These numbers show how much conversation and discussion now takes place on social media and that they have a significant influence on people.

Because of this high usage, massive amounts of data are created every day, which can be utilized through social media mining. Gundecha and Liu identified some key research issues to analyze social media: (1) community analysis, which deals with the detection of communities; (2) sentiment analysis, where opinions are extracted out of content; (3) social recommendation, which aims to recommend items based on the user's history and other users; (4) influence modeling, which discerns whether a community is driven by influence (certain key influencers) or homophily (similarity); (5) information diffusion and provenance, where the manner of information spread is analyzed; and (6) privacy, security, and trust, which grow more important due to the pervasive use of social media. \cite{Gundecha2012MiningSM}.

The number of possible research topics provided by social media is enormous. Sentiment analysis in particular is very interesting due to the inherent human interest in other people's opinions and viewpoints. Social media allows us to read and discuss opinions with a vast amount of people. Pang and Lee identified several applications for sentiment analysis. Reviews are very important to users, and automatic aggregation can be very valuable. Sentiment analysis also be used as a sub-component, for example in a recommendation system. In addition, business and government intelligence can benefit from sentiment analysis, for example, in reputation management and public relations, but also in product research \cite{pang-etal-2002-thumbs}. In politics especially, the analysis of population attitudes is very important, and sentiment analysis could be a supplement to existing polling methods, as shown by Brendan et al. \cite{polls}. 

Therefore, it is clear that sentiment analysis is a key part of the processing of social media data with a wide variety of applications. Most of the research for sentiment analysis can be classified into three main branches. The lexicon-based methods utilize sentiment lexicons which contain sentiment-bearing words with their polarity or polarity score. Machine learning methods employ machine learning classifiers, while hybrid methods combine the previous two approaches.

Sentiment analysis has been implemented on many different platforms, such as movie or product reviews. For social media, Facebook, MySpace and Twitter have often been used. Twitter differentiates itself from other domains due to its informal nature and length restrictions, which is why it is a very interesting research target. \cite{DBLP:journals/csur/GiachanouC16}.

The purpose of this thesis is to implement and evaluate all three approaches for Twitter sentiment analysis using existing data sets. Most research focuses only on one or two approaches and often utilizes different data sets to evaluate their methods, due to a lack of benchmark data sets and the difficulty of annotating Twitter data, which makes comparisons of different approaches more difficult \cite{DBLP:journals/csur/GiachanouC16}. Thus, this thesis allows for a better comparison between methods and other research papers.




\TODO{What does the reader already know --> duplicate info/adapt}



