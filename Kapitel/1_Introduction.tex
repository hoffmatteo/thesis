\chapter{Introduction}
\label{cha:Chapter1_Introduction}

\iffalse

Total length: up to 5 Months = ~20 weeks oder
\selectlanguage{german}{

Abstract fertig machen, konkreter werden --> vor allem Methoden, kleiner Ansatz


Length: 1-2 pages
Inhaltsverzeichnis --> auch mehr Punkte

Methodology and Implementation --> kann man auch anders aufteilen
Motivation, Hintergrund

Warum diese Plattform --> einfache API

Auch aufpassen illegal

Wann schreiben? --> am Anfang, am Ende? Erstmal Entwurf am Ende
}
\selectlanguage{english}
%
%Effort: 1-2 days
%\begin{itemize}
%\item Major importance of social media, impact on politics, source of information
%\item Massive number of posts allows for a good basis for analysis
%\item Analyse current trends, public's opinion on certain issues, news and other topics
%\item Motivation: Condense massive amount of information/opinions into an intelligible format
%\item Chosen platform: Twitter, short messages <280 characters, massive user base, broad topic range
%\end{itemize}
%

\fi
Kaplan and Haenlein define Social Media as "a group of Internet-based applications that build on the ideological and technological foundations of Web 2.0, and that allow the creation and exchange of User Generated Content" \cite[p.~61]{KAPLAN201059}. 

According to Pew Research Center, around 72\% of Americans use any kind of social media, with the most popular platforms being Facebook at 69\%, followed by Instagram at 40\% and Pinterest at 31\%. Different age groups also choose different platforms, as Snapchat is used by over half of 18-29 year olds, while only being used by 2\% of 65+ year olds. Twitter and Facebook on the other hand are the two platforms with the lowest difference between the youngest and oldest usage shares. Most platforms are also often used daily by their users, showing the importance in their daily lives \cite{pew:socialmedia}. Because of these usage numbers, massive amounts of data are created every day, which can and should be utilized through Social Media Mining. Gundecha and Liu identified some key challenges. Community Analysis deals with the detection of communities, Sentiment Analysis deals with the extraction of opinions out of content. Social Recommendation tries to recommend items based on the user's history and other users. Influence Modeling tries to discern whether a community is driven by influence (certain key influencers) or homophily (similarity). Information Diffusion and Provenance wants to analyze how information spreads in social media. Lastly, Privacy, Security, and Trust are very important as Social Media becomes more and more involved in personal life \cite{Gundecha2012MiningSM}.

The number of possible research topics provided by Social Media is enormous. Sentiment Analysis in particular is very interesting due to the inherent human interest in other people's opinions and viewpoints. Social Media allows us to read and discuss opinions with a vast amount of people. Pang and Lee identified several applications for Sentiment Analysis. Reviews are very important to users, and automatic aggregation can be very valuable. Sentiment Analysis also be used as a sub-component, for example in a recommendation system. In addition, business and government intelligence can benefit from sentiment analysis, for example, in reputation management and public relations, but also in product research \cite{pang-etal-2002-thumbs}. In politics especially, the analysis of population attitudes is very important, and Sentiment Analysis could be a supplement to existing polling methods, as shown by Brendan et al. \cite{polls}. 
\TODO{twitter!!}
Thus, it is clear that Sentiment Analysis is a key part of processing Social Media data with a wide variety of applications. Most of the research for Sentiment Analysis can be classified into three main branches. The lexicon-based methods utilize sentiment lexicons which contain sentiment-bearing with their polarity or polarity score. Machine-Learning methods employ machine-learning classifiers, while hybrid methods combine the previous two approaches. Approaches also often utilize different data sets to evaluate their methods, due to a lack of benchmark data sets and the difficulty of annotating Twitter data \cite{DBLP:journals/csur/GiachanouC16}.

The aim of this thesis is the implementation and evaluation of all three approaches using existing data sets. Most research focuses only on one or two approaches and often uses a new data set, which makes comparisons of different approaches more difficult. Thus, this thesis allows for easier categorization both inbetween methods and other research.




\TODO{What does the reader already know --> duplicate info/adapt}



