\chapter{Introduction}
\label{cha:Chapter1_Introduction}

\iffalse

Total length: up to 5 Months = ~20 weeks oder
\selectlanguage{german}{

Abstract fertig machen, konkreter werden --> vor allem Methoden, kleiner Ansatz


Length: 1-2 pages
Inhaltsverzeichnis --> auch mehr Punkte

Methodology and Implementation --> kann man auch anders aufteilen
Motivation, Hintergrund

Warum diese Plattform --> einfache API

Auch aufpassen illegal

Wann schreiben? --> am Anfang, am Ende? Erstmal Entwurf am Ende
}
\selectlanguage{english}
%
%Effort: 1-2 days
%\begin{itemize}
%\item Major importance of social media, impact on politics, source of information
%\item Massive number of posts allows for a good basis for analysis
%\item Analyse current trends, public's opinion on certain issues, news and other topics
%\item Motivation: Condense massive amount of information/opinions into an intelligible format
%\item Chosen platform: Twitter, short messages <280 characters, massive user base, broad topic range
%\end{itemize}
%

\fi

\TODO{figure out capilization of gauss, multinomial, etc...}

Social media has grown to be an important part of many people's daily lives. While many people know the most popular social media platforms such as Facebook, Twitter, and Instagram, Kaplan and Haenlein give a more specific definition. They define social media as "a group of Internet-based applications that build on the ideological and technological foundations of Web 2.0, and that allow the creation and exchange of User Generated Content" \cite[p.~61]{KAPLAN201059}. 

According to the Pew Research Center, about 72\% of Americans use any kind of social media, a number that is even higher if only young people are considered. Furthermore, a large number of users visit these platforms daily, sometimes even multiple times a day, showing how indispensable social media has become for many people \cite{pew:socialmedia}. The effect of this high usage can be seen in many areas of public and personal life. One interesting area is politics. In the 2016 U.S presidential election, the candidate's social media posts were more often used as a source of information compared to more traditional online platforms, such as their website and email \cite{pew:2016source}. In 2020, 23\% of U.S social media users stated that social media led them to change their opinion on an issue \cite{pew:2020influence}. These numbers show how much conversation and discussion now take place on social media and what a significant influence this has on people.

Because of this high usage, massive amounts of data are created every day, which can be utilized through social media mining. Gundecha and Liu identified some key research issues to analyze social media: (1) Community analysis, which deals with the detection of communities; (2) Sentiment analysis, where opinions are extracted out of content; (3) Social recommendation, which aims to recommend items based on the user's history and other similar users; (4) Influence modeling, which discerns whether a community is driven by influence (certain key influencers) or homophily (similarity); (5) Information diffusion and provenance, where the manner of information spread is analyzed; and (6) privacy, security, and trust, which grow more important due to the pervasive use of social media. \cite{Gundecha2012MiningSM}.
A large number of possible research topics are provided by social media. Sentiment analysis in particular is very interesting due to the inherent human interest in other people's opinions and viewpoints. In addition to this natural interest, Pang and Lee identified several applications for sentiment analysis. Reviews of products or movies are very important to users, and an automatic aggregation of reviews can be very valuable. Sentiment analysis could also be implemented as a subcomponent, for example, in a recommendation system. In addition, business and government intelligence can benefit from sentiment analysis, for example, in reputation management, public relations, and product research \cite{pang-etal-2002-thumbs}. Especially in politics, the analysis of population attitudes is very important, and sentiment analysis could be a supplement to existing polling methods, as shown by Brendan et al. \cite{polls}. 

Therefore, it is clear that sentiment analysis is a key part of the processing of social media data with a wide variety of applications. Most of the research for sentiment analysis can be classified into three main branches. The lexicon-based methods utilize sentiment lexicons, which contain sentiment-bearing words with their polarity. Machine learning methods employ machine learning classifiers, while hybrid methods combine the previous two approaches \cite{MEDHAT20141093}.

Sentiment analysis has been implemented on many different domains, such as movie or product reviews. In social media, Facebook, MySpace and Twitter have often been used. Twitter differentiates itself from other platforms due to its informal nature and length restrictions, which is why it is a very interesting research target. Through its API (application programming interface), Twitter also offers a method of collecting large amounts of data \cite{DBLP:journals/csur/GiachanouC16}.

The purpose of this thesis is to implement and evaluate the three main approaches for Twitter sentiment analysis using existing data sets. Most research articles focus only on one or two approaches and often utilize custom data sets to evaluate their methods, due to a lack of benchmark data sets and the difficulty in annotating Twitter data \cite{DBLP:journals/csur/GiachanouC16}. This makes evaluation of different approaches more difficult, which is why this thesis provides an easier comparison between the different methods.

This thesis is structured into 8 chapters. An introduction and the motivation for this thesis are given in chapter 1. Related work for the three different approaches to sentiment analysis is given in chapter 2, while chapter 3 outlines the fundamentals for data mining, sentiment analysis, and twitter. The methodology of each method and the evaluation metrics are explained in chapter 4, with the implementation and the data sets used being described in chapter 5. After this, chapter 6 presents the results and offers additional explanations for the findings. Finally, chapter 7 draws conclusions and offers an outlook for possible future improvements and topics. The code used in this thesis is given in chapter 8.




