\chapter{Methodology}
\label{cha:Chapter4_Methodology}

\section{Lexicon-based Method}

\subsection{Word classes}

\subsubsection{Sentiment words}
Sentiment words, also called opinion words, carry a positive or negative semantic orientation. Polarity-based lexicons only express whether a word is positive or negative, while a valence-based lexicon also determines the strength of a word. For example, a polarity-based lexicon would not differentiate between "bad" and "terrible", as both are negative, while a valence-based lexicon would establish "terrible" as having a stronger negative sentiment \cite{DBLP:conf/icwsm/HuttoG14}.

\subsubsection{Intensifiers}
\TODO{Quirk citation}

Taboada et al., according to Quirk et al., name two types of intensifiers, amplifiers and downtoners. While amplifiers such as "very" increase the \TODO{connected} sentiment, downtoners such as "barely" diminish it. A possible approach to implement an intensifier could be the addition or substraction based on its type. While this works, it doesn't take the connected sentiment into account. For example, "almost perfect" is much more positive than "almost good" and should scale accordingly. For this reason, an intensifier should be multiplicative \cite{10.1162/COLI_a_00049}.

\subsubsection{Negations}
Negations reverse the polarity of the connected sentiment, for example turning the positive sentiment "good" into the negative sentiment "not good". It is important though to note that negations do not necessarily appear right in front of the sentiment word they reverse, for example, an intensifier can be in between as in "not very good" \cite{10.1162/COLI_a_00049}.

\TODO{Source?, https://abs-0.twimg.com/emoji/v2/svg/1f600.svg}

\subsubsection{Emojis}
The Cambridge Dictionary defines an emoji as "a digital image that is added to a message in electronic communication in order to express a particular idea or feeling" \cite{cambridgeEmoji}. As Hu et al. found out, "the most popular intentions are expressing sentiment, strengthening expression, and adjusting tone" \cite[p.~109]{Hu_Guo_Sun_Nguyen_Luo_2017}. The use of emojis on Twitter has been steadily increasing, with 21.54\% of global tweets containing at least one emoji in the month of December 2021 \cite{emojiStatistic}. Thus, it is clear that emojis are an important part of a tweet's overall sentiment and need to be taken into consideration.







