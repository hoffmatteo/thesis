\chapter{Conclusion}
\label{cha:Chapter5_Conclusion}

\iffalse

Length: 1-2 pages

Effort: 1-2 days


Zusammenfassung der Ergebnisse, Ausblick --> was kann man noch machen, auf Ergebnisse aufbauen, weiterfuehrende Themen

Beim Erstellen der Arbeit --> nicht jeder Idee hinterherrennen, eher dann fuer Conclusion

Implementation source code --> muss nicht sein, kann auch ein link auf github sein

\fi

In this thesis, three different methods for Twitter sentiment analysis were discussed, implemented, and evaluated based on a set of defined measures. The lexicon-based method utilized multiple lexicons for specific word classes and achieved an accuracy of 71.04\%. For the machine learning method, several different classifiers and approaches were evaluated, with the Multinomial Naive Bayes, using word presence, unigrams, and bigrams, achieving the highest accuracy at 79.42\%. The hybrid approaches combined the previous two methods to balance their disadvantages, with the first approach using the lexicon-based score as an additional feature, and the second one labeling the training data with the lexicon-based method. The second method achieved the highest accuracy overall at 81.33\%, although both methods improved on the lexicon-based method and the machine learning method.

The advantages and disadvantages of each method were also discussed. The lexicon-based method was the least complex and resource-intensive, but also achieved the lowest accuracy. For use cases where performance is very important, such as analysis on large data sets, the accuracy is still high enough to be valuable, especially considering the domain-independence. If the goal is the highest accuracy possible, a machine learning-based classifier has to be involved. While the machine learning method requires more resources and is more complex, the accuracy is clearly superior. The hybrid approaches were able to improve the accuracy slightly, although the additional complexity may often not be worth it.

There are a few improvements and new methods that can be explored. For the lexicon-based method, there are two types of possible improvements. First, the rate of non-detected tweets can be improved from the current circa 10\%. The inclusion of certain phrases or idioms such as "can't wait" are usually not covered by sentiment lexicons, but still appear often in tweets. Hashtags may also be utilized more effectively, as Twitter users often combine multiple words into a single hashtag, for example, "\#itsNotCool". By not simply removing the hashtag character, but also trying to extract words, more tweets could be classified. Second, the correct classification of a detected sentiment can be improved, which accounts for circa 20\% of incorrect classifications. Here, the focus could be on detecting negative sentiment, as this had a higher error rate. Capitalization could be implemented, as a fully capitalized sentiment word could be amplified. Employing parts-of-speech tagging may also be important, as the sentiment of a word can change depending on its usage. For example, the word "break" has two different meanings depending on whether it is used as a verb or a noun ("to break something" compared to "to take a break"). Punctuation can also affect sentiment, for example, a stronger score could be assigned for sentences with one or multiple exclamation marks.


For machine learning methods, deep learning could be a further improvement, due to its different representation of documents. Deep learning employs neural networks to learn word features from a text and then apply those features to make predictions \cite{DBLP:journals/csur/GiachanouC16}. Giachanou and Crestani compiled a list of deep learning approaches for Twitter sentiment analysis, which were capable of achieving high accuracy \cite{DBLP:journals/csur/GiachanouC16}. In addition, more Twitter-specific features could be used. Metrics such as the number of likes or comments may be able to indicate the sentiment of the corresponding text.

Hybrid methods can also be further developed. Classifier ensembles, which combine one or multiple machine learning classifiers and a lexicon-based classifier by, for example, voting on the final result, have been shown to improve performance compared to a single classifier \cite{DBLP:journals/csur/GiachanouC16}.

Research into Twitter sentiment analysis has been steadily growing the past few years and has proven to be a valuable domain for sentiment analysis. Its combination of a wide variety of topics, short messages and large user base is very unique compared to traditional domains such as movie or product reviews. The interest in tracking sentiment and opinions on specific topics will only grow as social media platforms become larger and more influential, and research interest will continue to improve on the methods described in this thesis.

\TODO{licensing for data?}
\chapter{Implementation Code}
The implemented code, as well as the data used, can be found in the GitHub repository under the URL: \url{https://github.com/hoffmatteo/thesis_code}.
