\chapter{Methodology and Implementation}
\label{cha:Chapter4_Methodology}

Length: 20-25 pages

Effort: 8 weeks+

Teilung zwischen Methodik (eher abstrakt) und Implementierung
Damit beginnen --> also mit Implementierung

Questions:
\begin{itemize}
\item Structure --> differentiation between getting the data, analysing the data and evaluating the results ok? --> Kann man machen
\item General idea: ``Readers should be able to carry out the same procedure using the thesis'' --> level of detail, e.g. cloud services? --> Implementierung auch auf Github zur Verfügung stellen, API auf höherem Level (was machen die Funktionen verwenden),
Falls Cloud: nur virtueller Rechner --> eher nicht wichtig, falls spezifische Dienstleistung z.B. OpenAI (KI, Parsing) --> dann detaillierter, eher Richtung Implementation
\item Case study --> Evaluation of accuracy based on one real life topic? Topic, e.g. Covid-19 pandemic, german elections (aufpasssen, Politik schwer!), ...? --> je nach Umfang auch mehrere (sollen sich unterscheiden, Covid sehr viele Bereiche), Tweets auch auf Englisch
\end{itemize}

Content
\begin{itemize}
\item Data 
\begin{itemize}
    \item What data does twitter provide, auch z.B. Likes, Retweets
    \item Data extraction, Cloud Services --> spezifischer Dienst?
\end{itemize}
\item Analysis
\begin{itemize}
    \item Lexicon-Based Method
    \item Machine Learning Based Method
    \item Hybrid Method
\end{itemize}
\item Evaluation
\begin{itemize}
    \item Parameters --> objektive Kriterien, Qualität der Analyse Methode + Begründung der Kriterien, Komplexität (Laufzeitkomplexität ja, Implementierung --> an sich nicht schlimm, aber robust? Abstürze?)
    \item Case Study
    \item Qualität der Ergebnisse --> gut/schlecht, kann das akkurat erkannt werden?, evtl. menschliche Analyse, schöner: auf subjektive Einschätzung verzichten
\end{itemize}
\end{itemize}



