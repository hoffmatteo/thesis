\section{Hybrid Method}
\TODO{Sources!}

\subsection{Approach 1}
The first approach consisted of using the score of a lexicon-based method as an additional feature of the machine learning method. If a tweet is to be classified, it is first scored using the lexicon-based method. The calculated score is then used as an additional feature for the machine learning method. This has the advantage of not relying solely on the static dictionaries used by the lexicon-based method, while still being able to take specific sentence structures like negation into account.
\subsection{Approach 2}
The second approach trains the machine learning classifier using tweets labeled by the lexicon-based method. The precision of a machine learning classifier is based on its training data, both its quality and its size \cite{DBLP:journals/csur/GiachanouC16}. Because machine learning classifiers perform better with more training data, using human-labeled training data is too expensive. Some approaches like Go et al. classify tweets as positive or negative based on the presence of certain emoticons and are thus able to train using 1.6 million tweets \cite{GoBHaHua2009}. By annotating tweets using a more accurate lexicon-based approach, the quality of the training data could increase and thus the machine learning classifier could become more accurate.
\subsection{Approach 3}


