\chapter*{Abstract}
\selectlanguage{german}{Zusammenfassung der Bachelor-Arbeit auf Deutsch und Englisch. (Abstract)}

\selectlanguage{english}
Questions:
\begin{itemize}
\item How much explanation of the methods? --> nicht so viel, so wie ich es gemacht habe
\item Should extraction of tweets be part of the abstract?
\item Should the results be part of the abstract? --> rauslassen, eher Motivation --> contributions Werbung nicht Kurzzusammenfassung,Ziele
\item Motivation/introduction?
Ziele: Was genau soll ermittelt werden, was machen mit diesen Ergebnissen --> results can be used to implement means of evaluating --> Praxisbezug, ein Satz reicht
\end{itemize}

\TODO{weiter machen}
The increasing popularity of social media provides new data sources to determine the public's opinion using sentiment analysis. The aim of this thesis is the evaluation and comparison of three different approaches to this topic.

The first approach is lexicon-based, which determines the overall sentiment by assessing each word using the scores defined by the dictionary. The second method employs a supervised machine learning classifier that is trained on labelled tweets. The last approach combines the previous two in order to overcome some of their drawbacks, such as the intrinsic limitation of using a finite lexicon or the need to provide tweets for training. Each method is executed and then evaluated based on a defined set of parameters using a case study on <insert topic>.

The results show that ...





