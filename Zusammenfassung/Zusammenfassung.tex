\chapter*{Abstract}

\iffalse

\selectlanguage{english}
Questions:
\begin{itemize}
\item How much explanation of the methods? --> nicht so viel, so wie ich es gemacht habe
\item Should extraction of tweets be part of the abstract?
\item Should the results be part of the abstract? --> rauslassen, eher Motivation --> contributions Werbung nicht Kurzzusammenfassung,Ziele
\item Motivation/introduction?
Ziele: Was genau soll ermittelt werden, was machen mit diesen Ergebnissen --> results can be used to implement means of evaluating --> Praxisbezug, ein Satz reicht
\end{itemize}
\fi

The increasing popularity of social media provides new data sources to determine the public's opinion using sentiment analysis. Especially Twitter is an interesting platform due to its focus on short messages and its wide influence. The aim of this thesis is the implementation and evaluation of three different approaches to Twitter sentiment analysis. The first approach is lexicon-based, which determines the overall sentiment of a tweet by assessing each sentiment word using a dictionary. The second method employs a supervised machine learning classifier that is trained on labeled tweets. The last approach combines the previous two in order to overcome some of their drawbacks, such as the intrinsic limitation of using a finite lexicon or the need to provide tweets for training. Each method is implemented and then evaluated based on a defined set of parameters, using two existing human labeled data sets. The results of each method, as well as the advantages and disadvantages are then discussed.


\selectlanguage{german}
Die zunehmende Beliebtheit sozialer Medien ermöglicht neue Datenquellen, die mittels Sentimentanalyse zur Bestimmung der öffentlichen Meinung benutzt werden können. Insbesondere Twitter ist dabei eine sehr interessante Plattform, da diese einen großen Einfluss hat und sich auf kurze Nachrichten fokussiert. Das Ziel dieser Bachelorarbeit ist die Implementierung und Evaluierung drei verschiedener Ansätze für Sentimentanalyse auf Twitter. Der erste Ansatz ist lexikon-basiert, und ermittelt die Gesamtstimmung eines Tweets, indem es jedes Stimmungswort mithilfe eines Wörterbuchs bewertet. Der zweite Ansatz benutzt einen überwachten Machine Learning Klassifikator, der mithilfe von bereits gewerteten Tweets trainiert wird. Der letzte Ansatz kombiniert die beiden vorherigen Ansätze, um ihre Nachteile auszugleichen, wie z.B. die Benutzung eines endlichen Wörterbuchs oder die Notwendigkeit von bereits gewerteten Tweets. Jeder Ansatz wird implementiert und anhand von definierten Metriken und zwei bestehender, von Menschen gewerteten Datensätzen evaluiert. Die Ergebnisse jeder Methode sowie die jeweiligen Vor- und Nachteile werden anschließend diskutiert.



\selectlanguage{english}









