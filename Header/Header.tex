\documentclass[12pt, oneside, a4paper, bibgerm, numbers=noenddot, parskip=half]{scrreprt}
\usepackage[T1]{fontenc} 

\headheight 50pt

\setcounter{secnumdepth}{3}
\setcounter{tocdepth}{3}


\usepackage[german, english]{babel}
\usepackage[T1]{fontenc}
\usepackage[macce]{inputenc}

\usepackage{graphicx}
\usepackage{placeins}
\usepackage{subfig}
\usepackage{bibgerm}
\usepackage{url}
\bibliographystyle{alphadin}

\usepackage{algorithm}
\usepackage{algpseudocode}
\algrenewcommand\algorithmicrequire{\textbf{Input:}}
\algrenewcommand\algorithmicensure{\textbf{Output:}}

\renewcommand{\familydefault}{\sfdefault}
\renewcommand{\baselinestretch}{1.0}

\makeatletter
\renewcommand\l@figure{\@dottedtocline{1}{0em}{2.6em}}
\renewcommand\l@table{\@dottedtocline{1}{0em}{2.6em}}
\makeatother


\def\mydotfill{
\leaders\hbox to 0.80em{.\hss}\hfill}
\usepackage{nomencl}
\let\abk\nomenclature
\renewcommand{\nomname}{AbkŸrzungsverzeichnis}
\renewcommand{\nompreamble}{\vspace*{0.6\baselineskip}\markboth{\nomname}{\nomname}}
\setlength{\nomlabelwidth}{0.4\textwidth}				% Setzt den Abstand zwischen AbkŸrzung und ErklŠrung
\renewcommand{\nomlabel}[1]{#1 \mydotfill}			% Punkte zwischen AbkŸrzung und ErklŠrung
\setlength{\nomitemsep}{-\parsep}					% ZeilenabstŠnde zwischen den AbkŸrzungen verkleinern
\usepackage[normalem]{ulem}
\newcommand{\markup}[1]{\uline{#1}}
\makenomenclature

\setlength{\parindent}{0pt}

\usepackage{listings}
\usepackage{color}

\definecolor{R.red}{RGB}{187,0,0}
\definecolor{R.green}{RGB}{127,166,0}

\lstset{language=Perl,
	 basicstyle=\small,
	 numbers=left,
	 numberstyle=\small,
	 numbersep=5pt,
	 captionpos=b,
	 keywordstyle=\color{R.red},
	 commentstyle=\color{R.green}
}

\renewcommand{\labelitemi}{--}

\usepackage[percent]{overpic}
\pdfminorversion=5
\usepackage[absolute]{textpos}

\usepackage{ntheorem} 
\usepackage{mathtools} 


\usepackage{float}
\restylefloat{figure}

\usepackage{booktabs}
\usepackage{tabularx}
\usepackage{supertabular}
\usepackage{multirow}
\usepackage{multicol}

\pagestyle{headings}

\lstdefinelanguage{Lua}{%
morekeywords={%
and,break,do,else,elseif,end,false,for,function,if,in,%
local,nil,not,or,repeat,return,then,true,until,while%
},%
sensitive=true,%
morecomment=[l]{--},%
morecomment=[n]{--[[}{]]},%
morestring=[b]",%
morestring=[b]',%
}[keywords,comments,strings]% 

\newcommand{\tc}[1]{\ensuremath{\underline{\smash{\mathit{#1}}}}}

\DeclareMathOperator{\atan2}{atan2}

